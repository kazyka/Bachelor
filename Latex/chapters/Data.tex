\chapter{Data}

The MRI head scans were acquired on a General Electric 3-T for all our 100 patients, were we got 50 normal subjects and 50 AD patients.

ADNI data
The data in this study was provided already downloaded and preprocessed. It had been previously obtained from the ADNI database (adni.loni.usc.edu). The ADNI was launched in 2003 by the National Institute on Aging, the National Institute of Biomedical Imaging and Bioengineering, the Food and Drug Administration, private pharmaceutical companies, and nonprofit organizations, as a \$60 million, 5-year, public–private partnership. The primary goal of ADNI has been to test whether serial MRI, positron emission tomography (PET), biological markers, and clinical and neuro-psychological assessments can be combined to measure the progression of MCI and early AD. Determination of sensitive and specific markers of very early AD progression is intended to aid researchers and clinicians to develop new treatments and monitor their effectiveness, as well as to lessen the time and cost of clinical trials. ADNI is the result of efforts of many co-investigators from a broad range of academic institutions and private corporations, and subjects have been recruited from over 50 sites across the U.S. and Canada. For up-to-date information, see http://www.adni-info.org/.
We were provided with a random subset (50 controls, 50 AD) of the “complete annual year 2 visits” 1.5T dataset from the collection of standardized datasets released by ADNI (http://www.adni.loni.usc.edu/methods/mri-analysis/adni-standardized-data/) [Wyman et al., 2013]. The complete dataset (504 subjects) comprised one associated 1.5T T1-weighted MRI image out of the two possible from the back-to-back scanning protocol in ADNI [Jack et al., 2008] at baseline, 12-month follow-up, and 24-month follow-up.
Jack J, Clifford R, Bernstein MA, Fox NC, Thompson P, Alexander G, Harvey D, Borowski B, Britson PJL, Whitwell J, Ward C, et al. (2008): The Alzheimer’s Disease Neuroimaging Initiative (ADNI): MRI methods. J Magn Reson Imaging 27:685–691.
Wyman BT, Harvey DJ, Crawford K, Bernstein MA, Carmichael O, Cole PE, Crane PK, Decarli C, Fox NC, Gunter JL, et al. (2013): Standardization of analysis sets for reporting results from ADNI MRI data. Alzheimers Dement 9:332–337.

Preprocessing
The data were provided for use already preprocessed. This preprocessing, and subsequent hippocampal segmentation was performed with the freely available FreeSurfer software package (version 5.1.0) [Fischl et al., 2002] using the cross-sectional pipeline with default parameters. The original MRI image resolution of [0.94, 1.35] x [0.94, 1.35] x 1.2mm was conformed to a 1.0 x 1.0 x 1.0 mm resolution, and all MRIs were bias field corrected. The bias field correction in FreeSurfer utilizes the nonparametric nonuniform intensity normalization algorithm [Sled et al., 1998], often referred to as N3. The input data for this study was therefore the corrected T1-weighted MRI image volume, and corresponding separate binary masks of left and right hippocampi.
Fischl B, Salat DH, Busa E, Albert M, Dieterich M, Haselgrove C, van der Kouwe A, Killiany R, Kennedy D, Klaveness S, et al. (2002): Whole brain segmentation: automated labeling of neuro- anatomical structures in the human brain. Neuron 33:341–355.
Sled JG, Zijdenbos AP, Evans AC (1998): A nonparametric method for automatic correction of intensity nonuniformity in MRI data. IEEE Trans Med Imaging 17:87–97.