\chapter{Introduction}

Alzheimer's Disease (AD) is the most common cause of dementia among people and is a growing problem in the aging populations. It has a big impact on health services and society as life expectancy increases. In 2010 the total global costs of dementia was estimated to be about 1\% of the worldwide gross domestic product\footnote{With the terms as of direct medical costs, direct social costs and costs of informal care}. AD is the cause in about 60\%-70\% of all cases of dementia\cite{Who} and about 70\% of the risk is believed to be genetic \cite{AlzheimerLancet}. Currently there are no way to cure dementia or to alter the progressive course. But however, much can be done to support and improve the lives if AD is diagnosed in the early stage of progression \cite{Who}

In this report we will examine MRI data of the hippocampus using image texture analysis and apply machine learning in order to diagnose AD in patients, and attempt to achieve a accuracy higher than 80\%. Our data set contain 100 patients, of 50 control and 50 with AD.

We will be using two different image texture analysis method, one which will be similar to the one used in 2D\cite{MRfreeborough}\cite{Castellano} and the other where we attempt to extend it into three-dimensions, from which we calculate the data to the gray level co-oc\-curren\-ce matrix (GLCM). We will evaluate the two methods to see if their is a significant difference in performance.

\section{Problem Definition}

Is it possible to classify MRI data of the hippocampus into groups of healthy controls vs Alzheimer's patient, using a predefined set of image texture features, with an accuracy great\-er than 80\%?

Is there a difference in diagnosing AD successfully by calculating the co-occurrence matrix in 3D compared to 2D?


