\chapter{Introduction}

In this report we will examine MRI data of the hippocampus using image texture analysis and apply machine learning. We have XX normal controls and XX Alzheimer's Disease (AD) patients. They are split into a training set (XX control and XX AD) and a test set (XX control and XX AD).

We will be using two different texture analysis on the data, XX and the gray level co-occ\-urren\-ce matrix (GLCM).
We will calculate the GLCM using two different methods, one in 2D that that runs along the z-axis with angel 90 and distance 1 (Change depending on the results, and more research might include multiple angles).\cite{Castellano}
The other method is calles voxel-based GLCM in 3D\fxnote[inline]{citation} space (VGLCM-TOP-3D), and is from the paper Voxel-Based Texture Analysis of the Brain (indsæt biblografi).
We want to see if there is a difference between diagnostising a AD succesfully, by calculating the co-occurence matrix in 3D compared to 2D, and how well the GLCM methods work
compared to XX. To do that we will use two different machine learning methods, k-NN and Gaussian mixture, based on each of the image texture models.

We will also try to replicate the analysis from \cite{MRfreeborough} and meanwhile tell if we can get better accuracy

%We will focus on a co occurrence matrix both in 2D and 3D and k-NN as machine learning. If we have time, we will try to implement more methods\footnote{\url{http://etd.fcla.edu/CF/CFE0000273/Gadkari_Dhanashree_U_200412_MS.pdf}}

\section{Problem Definition}

Is it possible to classify MRI data of the hippocampus into groups of healthy controls vs Alzheimer's patient, using a predefined set of image texture metrices, with an accuracy great\-er than 80\%?\fxnote[inline]{Mere i intro. Måske 3D vs 2D}

\section{Alzheimer's Disease}

About 70\% of the risk is believed to be genetic, where other factors include head injuries, depression or hypertension. \cite{Magnetic} \fxnote[inline]{Mere til AD}




