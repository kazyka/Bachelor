\chapter{Discussion}

In this paper when calculating the GLCMs we have chosen to work with offsets similar to those used in the paper by Peter A. Freeborough and Nick C. Fox \cite{MRfreeborough} in contrast to the radius method used by Rouzbeh Maani, Yee Hong Yang, Sanjay Kalra \cite{Voxel}.
The reason we chose to work with offsets is because we wished to compare our results with \cite{MRfreeborough}


lille intro :)

Start med at snakke om Normalized Erode

Sammenligne med 2D og 3D

Hvorfor er 2D bedre end 3D eller omvendt. hvilke angles er i virkeligheden forskellige?

Overfitting

Diskutere valg af offsets og Offsets vs radius



På baggrund af graferne hvorfor det kan være svært at få gode res.

\section{GOD SECTION TITEL}
In the previous chapter we saw the plots of our data, which are early AD patient, more specific 24-month follow-ups. As our data consists of early AD patients, it can be very difficult to differentiate one from another and thus make it challenging to get some good results, specially if we are to select some features to do machine learning. But luckily we can lean on our algorithm to select features better than we are. In consideration of that the data is so

FFS vs naive selection


Hvorfor, hvorfor ikke normalisere.

Diskutere Erode vs Ikke Erode

Snakke om 60/40

Plots af breaks i accuracy og ændring i k for knn 