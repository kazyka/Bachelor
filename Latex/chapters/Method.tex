\chapter{Method}



\section{Image texture analysis methods}

\subsection{Co occurrence matrix}

Co occurence matrix is defined over an image to be the distribution values at a given offset.\cite{Bharati} With the co occurence matrix, we have matrix \textbf{C} defined over an $n \times m$ image, with $\Delta x, \Delta y$ being the parameterized offset, so

\[
C_{\Delta x, \Delta y}(i, j) = \Sigma_{p=1}^n\Sigma_{q=1}^m
\begin{dcases}
  1, \quad \text{if } I(p,q)=i \text{ and } I(p+\Delta x, q+ \Delta y) =j \\
  0, \quad \text{otherwise}
\end{dcases}
\]

\section{Machine learning methods}

\subsection{K-nearest neighbors algorithm}

k-NN for short is a method that is used for classification and regression. Where the output is a class and member of this class, and this object is classified by its neighbors. For instance, if we chose k to 1, then the object will be assigned to the class of the single nearest neighbor.

The algorithm consist of training examples, that are vectors in multidimensional space, with each its label. The most used distance metric is Euclidean distance.

The drawback of k-NN is that classification can be skewed in that way, that the more frequent class tend to dominate the prediction of new examples, because they tend to be common among the k-NN due to their large number.

The way we wish to implement the k-NN in matlab is, first we handle the data, then we will calculate the distance between two data instances and after that, we can locate k most similar data instances and generate a response from a set. After all this is done, we have to summarize the accuracy of predictions.


Dette vil være en lille introduktion til de tools vi bruger til at analysere vores data med. Da vores data er MRI skanninger af hjernen, som er nogen voxels\footnote{noget her} som bliver repræsenteret i 3D.\\



\section{Erode}
Normalt brugte man dette til binære billeder, men senere hen er det udvidet til også at omfatte grayscale billeder. Grunden til at dette bruges er for at fjerne støj på billedet.

Vi bruger Erode på vores MRI scanning, da der kan være gray-bit mix og derfor fjerner overflydig og blandet data. Hvis vi starter på hvordan 2D virker, så

For at illustrere hvordan erosion virker, gør vi det på et 2D plan, betragt figur \ref{Erosion2D}, hvor vi bruger et plus til at fjerne støj.
\begin{figure}[H]
  \centering
    \begin{tikzpicture}[scale=0.5]
    \matrix[square matrix]
    {
    |[fill=white]| & |[fill=white]|   & |[fill=white]|   & |[fill=white]|   & |[fill=white]|\\
    |[fill=white]| & |[fill=white]|   & |[fill=yellow!30]|$N_2$ & |[fill=white]|   & |[fill=white]|\\
    |[fill=white]| & |[fill=yellow!30]|$N_1$ & |[fill=yellow!30]|$\boxplus$ & |[fill=yellow!30]|$N_3$ & |[fill=white]|\\
    |[fill=white]| & |[fill=white]|   & |[fill=yellow!30]|$N_4$ & |[fill=white]|   & |[fill=white]|\\
    |[fill=white]| & |[fill=white]|   & |[fill=white]|   & |[fill=white]|   & |[fill=white]| \\
    };
    \end{tikzpicture}
  \caption{Text}\label{Erosion2D}
\end{figure}
Så med figur \ref{Erosion2D}, bruger vi denne på figur \ref{ErosionExample}.
\begin{figure}[H]
  \centering
    \begin{tikzpicture}
    \matrix[square matrixlarge]
    {
    |[fill=white]| & |[fill=white]| & |[fill=white]| & |[fill=white]| & |[fill=white]| & |[fill=white]| & |[fill=white]| & |[fill=white]| & |[fill=white]| & |[fill=white]| \\
    |[fill=white]| & |[fill=white]| & |[fill=white]| & |[fill=white]| & |[fill=white]| & |[fill=white]| & |[fill=white]| & |[fill=white]| & |[fill=white]| & |[fill=white]| \\
    |[fill=white]| & |[fill=white]| & |[fill=white]| & |[fill=blue!30]| & |[fill=blue!30]| & |[fill=white]| & |[fill=white]| & |[fill=white]| & |[fill=white]| & |[fill=white]| \\
    |[fill=white]| & |[fill=white]| & |[fill=blue!30]| & |[fill=blue!30]| & |[fill=blue!30]| & |[fill=blue!30]| & |[fill=white]| & |[fill=white]| & |[fill=white]| & |[fill=white]| \\
    |[fill=white]| & |[fill=white]| & |[fill=blue!30]| & |[fill=blue!30]| & |[fill=blue!30]| & |[fill=blue!30]| & |[fill=white]| & |[fill=white]| & |[fill=white]| & |[fill=white]| \\
    |[fill=white]| & |[fill=white]| & |[fill=white]| & |[fill=blue!30]| & |[fill=blue!30]| & |[fill=blue!30]| & |[fill=blue!30]| & |[fill=white]| & |[fill=white]| & |[fill=white]| \\
    |[fill=white]| & |[fill=white]| & |[fill=white]| & |[fill=blue!30]| & |[fill=blue!30]| & |[fill=blue!30]| & |[fill=blue!30]| & |[fill=white]| & |[fill=white]| & |[fill=white]| \\
    |[fill=white]| & |[fill=white]| & |[fill=white]| & |[fill=white]| & |[fill=white]| & |[fill=blue!30]| & |[fill=white]| & |[fill=white]| & |[fill=white]| & |[fill=white]| \\
    |[fill=white]| & |[fill=white]| & |[fill=white]| & |[fill=white]| & |[fill=white]| & |[fill=white]| & |[fill=white]| & |[fill=white]| & |[fill=white]| & |[fill=white]| \\
    |[fill=white]| & |[fill=white]| & |[fill=white]| & |[fill=white]| & |[fill=white]| & |[fill=white]| & |[fill=white]| & |[fill=white]| & |[fill=white]| & |[fill=white]| \\
    };
    \end{tikzpicture}
    \begin{tikzpicture}
    \matrix[square matrixlarge]
    {
    |[fill=white]| & |[fill=white]| & |[fill=white]| & |[fill=white]| & |[fill=white]| & |[fill=white]| & |[fill=white]| & |[fill=white]| & |[fill=white]| & |[fill=white]| \\
    |[fill=white]| & |[fill=white]| & |[fill=white]| & |[fill=white]| & |[fill=white]| & |[fill=white]| & |[fill=white]| & |[fill=white]| & |[fill=white]| & |[fill=white]| \\
    |[fill=white]| & |[fill=white]| & |[fill=white]| & |[fill=yellow!50]| & |[fill=yellow!50]| & |[fill=white]| & |[fill=white]| & |[fill=white]| & |[fill=white]| & |[fill=white]| \\
    |[fill=white]| & |[fill=white]| & |[fill=yellow!50]| & |[fill=blue!30]| & |[fill=blue!30]| & |[fill=yellow!50]| & |[fill=white]| & |[fill=white]| & |[fill=white]| & |[fill=white]| \\
    |[fill=white]| & |[fill=white]| & |[fill=yellow!50]| & |[fill=blue!30]| & |[fill=blue!30]| & |[fill=yellow!50]| & |[fill=white]| & |[fill=white]| & |[fill=white]| & |[fill=white]| \\
    |[fill=white]| & |[fill=white]| & |[fill=white]| & |[fill=yellow!50]| & |[fill=blue!30]| & |[fill=blue!30]| & |[fill=yellow!50]| & |[fill=white]| & |[fill=white]| & |[fill=white]| \\
    |[fill=white]| & |[fill=white]| & |[fill=white]| & |[fill=yellow!50]| & |[fill=yellow!50]| & |[fill=blue!30]| & |[fill=yellow!50]| & |[fill=white]| & |[fill=white]| & |[fill=white]| \\
    |[fill=white]| & |[fill=white]| & |[fill=white]| & |[fill=white]| & |[fill=white]| & |[fill=yellow!50]| & |[fill=white]| & |[fill=white]| & |[fill=white]| & |[fill=white]| \\
    |[fill=white]| & |[fill=white]| & |[fill=white]| & |[fill=white]| & |[fill=white]| & |[fill=white]| & |[fill=white]| & |[fill=white]| & |[fill=white]| & |[fill=white]| \\
    |[fill=white]| & |[fill=white]| & |[fill=white]| & |[fill=white]| & |[fill=white]| & |[fill=white]| & |[fill=white]| & |[fill=white]| & |[fill=white]| & |[fill=white]| \\
    };
    \end{tikzpicture}
    \begin{tikzpicture}
    \matrix[square matrixlarge]
    {
    |[fill=white]| & |[fill=white]| & |[fill=white]| & |[fill=white]| & |[fill=white]| & |[fill=white]| & |[fill=white]| & |[fill=white]| & |[fill=white]| & |[fill=white]| \\
    |[fill=white]| & |[fill=white]| & |[fill=white]| & |[fill=white]| & |[fill=white]| & |[fill=white]| & |[fill=white]| & |[fill=white]| & |[fill=white]| & |[fill=white]| \\
    |[fill=white]| & |[fill=white]| & |[fill=white]| & |[fill=white]| & |[fill=white]| & |[fill=white]| & |[fill=white]| & |[fill=white]| & |[fill=white]| & |[fill=white]| \\
    |[fill=white]| & |[fill=white]| & |[fill=white]| & |[fill=blue!30]| & |[fill=blue!30]| & |[fill=white]| & |[fill=white]| & |[fill=white]| & |[fill=white]| & |[fill=white]| \\
    |[fill=white]| & |[fill=white]| & |[fill=white]| & |[fill=blue!30]| & |[fill=blue!30]| & |[fill=white]| & |[fill=white]| & |[fill=white]| & |[fill=white]| & |[fill=white]| \\
    |[fill=white]| & |[fill=white]| & |[fill=white]| & |[fill=white]| & |[fill=blue!30]| & |[fill=blue!30]| & |[fill=white]| & |[fill=white]| & |[fill=white]| & |[fill=white]| \\
    |[fill=white]| & |[fill=white]| & |[fill=white]| & |[fill=white]| & |[fill=white]| & |[fill=blue!30]| & |[fill=white]| & |[fill=white]| & |[fill=white]| & |[fill=white]| \\
    |[fill=white]| & |[fill=white]| & |[fill=white]| & |[fill=white]| & |[fill=white]| & |[fill=white]| & |[fill=white]| & |[fill=white]| & |[fill=white]| & |[fill=white]| \\
    |[fill=white]| & |[fill=white]| & |[fill=white]| & |[fill=white]| & |[fill=white]| & |[fill=white]| & |[fill=white]| & |[fill=white]| & |[fill=white]| & |[fill=white]| \\
    |[fill=white]| & |[fill=white]| & |[fill=white]| & |[fill=white]| & |[fill=white]| & |[fill=white]| & |[fill=white]| & |[fill=white]| & |[fill=white]| & |[fill=white]| \\
    };
    \end{tikzpicture}
  \caption{Left: Middle: Right:}\label{ErosionExample}
\end{figure}
Således er støjen nu fjernet. Vi vil udvide dette til 3D, da vores MRI er i 3D. Som det ses på figur \ref{Erosion2D} har denne 4 naboer den tjekker, når man udvider til 3D, får vi 2 nye naboer, dvs 6 naboer i alt. Hvis en af dem er udenfor den ønskede matrix, eksluderes pixlen.


Da vores data er i 3D, så i udvider vi erosion, hvor det stadig er et plus, men med 2 ekstra naboer
Now we expand this cross for the 3D and its the same concept for 3D. Now it 6 neighbours instead of 4, where we expand it for the 3D
\begin{figure}[H]
  \centering
    \begin{tikzpicture}
    \matrix[square matrix]
    {
    |[fill=white]| & |[fill=white]|   & |[fill=white]|   & |[fill=white]|   & |[fill=white]|\\
    |[fill=white]| & |[fill=white]|   & |[fill=white]| & |[fill=white]|   & |[fill=white]|\\
    |[fill=white]| & |[fill=white]| & |[fill=yellow!30]|$N_5$ & |[fill=white]| & |[fill=white]|\\
    |[fill=white]| & |[fill=white]|   & |[fill=white]| & |[fill=white]|   & |[fill=white]|\\
    |[fill=white]| & |[fill=white]|   & |[fill=white]|   & |[fill=white]|   & |[fill=white]| \\
    };
    \end{tikzpicture}
    \begin{tikzpicture}
    \matrix[square matrix]
    {
    |[fill=white]| & |[fill=white]|   & |[fill=white]|   & |[fill=white]|   & |[fill=white]|\\
    |[fill=white]| & |[fill=white]|   & |[fill=yellow!30]|$N_2$ & |[fill=white]|   & |[fill=white]|\\
    |[fill=white]| & |[fill=yellow!30]|$N_1$ & |[fill=yellow!30]|$\boxplus$ & |[fill=yellow!30]|$N_3$ & |[fill=white]|\\
    |[fill=white]| & |[fill=white]|   & |[fill=yellow!30]|$N_4$ & |[fill=white]|   & |[fill=white]|\\
    |[fill=white]| & |[fill=white]|   & |[fill=white]|   & |[fill=white]|   & |[fill=white]| \\
    };
    \end{tikzpicture}
    \begin{tikzpicture}
    \matrix[square matrix]
    {
    |[fill=white]| & |[fill=white]|   & |[fill=white]|   & |[fill=white]|   & |[fill=white]|\\
    |[fill=white]| & |[fill=white]|   & |[fill=white]| & |[fill=white]|   & |[fill=white]|\\
    |[fill=white]| & |[fill=white]| & |[fill=yellow!30]|$N_6$ & |[fill=white]| & |[fill=white]|\\
    |[fill=white]| & |[fill=white]|   & |[fill=white]| & |[fill=white]|   & |[fill=white]|\\
    |[fill=white]| & |[fill=white]|   & |[fill=white]|   & |[fill=white]|   & |[fill=white]| \\
    };
    \end{tikzpicture}
  \caption{Text}\label{Erosion3D}
\end{figure}

\section{Image texture (Co-occurrence)}

Et image texture er bare et sæt af metricer? som udregnes for at opfatte texturen på et billede. Image textures giver os denne information.

Der findes flere approaches til hvordan man udregner image textures, vi bruger den statistiske vej. Helt specifikt udregner vi en co occurence matricer, hvor man kan få en del numeriske features fra gray tones. Disse kan ses i appendix \ref{derivationfeatures} på s. \pageref{derivationfeatures}.

Hvordan vi har 256x256 matrix, med en distance på $\delta$ og angles $\theta$, hvor $\delta = \{1,2,...,10\}$ og $\theta = \{0^\circ, 45^\circ, 90^\circ, 135^\circ\}$.

Ud fra denne GLCM matrix så kan vi udregne textural features fra de 100 GLCM af MRI grayscale image data set.

Snakke om hvordan den fungere, med distance og grader

\section{Principal Component analysis}

\subsection{Application to images}

\section{K-nearest neighbors}

\subsection{Cross validation}




Image texture
PCA
Principal Component Analysis Application to images
Machine learning (Knn, Ann, Gaussian)


\small
noget tekst \fxfatal{her er noget galt}

noget mere tekst \fxnote[inline]{en note}

endnu mere tekst \fxwarning{en advarsel -- noget er helt forkert}

og til slut mere tekst \fxerror[inline]{en fejl}
