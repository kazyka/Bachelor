\chapter{Intro}

In this report we will examine MRI data of the hippocampus using image texture analysis and apply machine learning. We have XX normal controls and XX Alzheimer's Disease (AD) patients. They are split into a training set (XX control and XX AD) and a test set (XX control and XX AD).

We will be using two different texture analysis on the data, XX and the gray level co-occ\-urren\-ce matrix (GLCM).
We will calculate the GLCM using two different methods, one in 2D that that runs along the z-axis with angel 90 and distance 1 (Change depending on the results, and more research might include multiple angles).\cite{Castellano}
The other method is calles voxel-based GLCM in 3D\fxnote{citation} space (VGLCM-TOP-3D), and is from the paper Voxel-Based Texture Analysis of the Brain (indsæt biblografi).
We want to see if there is a difference between diagnostising a AD succesfully, by calculating the co-occurence matrix in 3D compared to 2D, and how well the GLCM methods work
compared to XX. To do that we will use two different machine learning methods, k-NN and Gaussian mixture, based on each of the image texture models.

We will also try to replicate the analysis from \cite{MRfreeborough} and meanwhile tell if we can get better accuracy

%We will focus on a co occurrence matrix both in 2D and 3D and k-NN as machine learning. If we have time, we will try to implement more methods\footnote{\url{http://etd.fcla.edu/CF/CFE0000273/Gadkari_Dhanashree_U_200412_MS.pdf}}

\section{Problem Definition}

Is it possible to classify MRI data of the hippocampus into groups of healthy controls vs Alzheimer's patient, using a predefined set of image texture metrices, with an accuracy great\-er than 80\%?

\section{Alzheimer's Disease}

About 70\% of the risk is believed to be genetic, where other factors include head injuries, depression or hypertension. \cite{Magnetic}

\section{Image texture analysis methods}

\subsection{Co occurrence matrix}

Co occurence matrix is defined over an image to be the distribution values at a given offset.\cite{Bharati} With the co occurence matrix, we have matrix \textbf{C} defined over an $n \times m$ image, with $\Delta x, \Delta y$ being the parameterized offset, so

\[
C_{\Delta x, \Delta y}(i, j) = \Sigma_{p=1}^n\Sigma_{q=1}^m
\begin{dcases}
  1, \quad \text{if } I(p,q)=i \text{ and } I(p+\Delta x, q+ \Delta y) =j \\
  0, \quad \text{otherwise}
\end{dcases}
\]

\section{Machine learning methods}

\subsection{K-nearest neighbors algorithm}

k-NN for short is a method that is used for classification and regression. Where the output is a class and member of this class, and this object is classified by its neighbors. For instance, if we chose k to 1, then the object will be assigned to the class of the single nearest neighbor.

The algorithm consist of training examples, that are vectors in multidimensional space, with each its label. The most used distance metric is Euclidean distance.

The drawback of k-NN is that classification can be skewed in that way, that the more frequent class tend to dominate the prediction of new examples, because they tend to be common among the k-NN due to their large number.

The way we wish to implement the k-NN in matlab is, first we handle the data, then we will calculate the distance between two data instances and after that, we can locate k most similar data instances and generate a response from a set. After all this is done, we have to summarize the accuracy of predictions.



