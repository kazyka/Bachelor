\chapter{Analysis of the data}

%\begin{table}
%  \centering
%  \begin{tabular}{|c|c|c|c|}
%    \hline
%    % after \\: \hline or \cline{col1-col2} \cline{col3-col4} ...
%    Parameter & GLCM Feature & Mean/Range & Distance \\
%    a & Angular Second Moment &  &  \\
%    b & Contrast &  &  \\
%    c & Correlation &  &  \\
%    d & Variance &  &  \\
%    e & Sum Average &  &  \\
%    f & Sum Variance &  &  \\
%    g & Sum Entropy &  &  \\
%    h & Entropy &  &  \\
%    i & Difference Variance &  &  \\
%    j & Difference Entropy &  &  \\
%    k & Inverse Difference Moment &  &  \\
%    l &  &  &  \\
%    m &  &  &  \\
%    n &  &  &  \\
%    o &  &  &  \\
%    p &  &  &  \\
%    q &  &  &  \\
%     &  &  &  \\
%     &  &  &  \\
%    \hline
%  \end{tabular}
%  \caption{}\label{}
%\end{table}


Nogen angle og planes er ens

Når vi loader ind i en train og test, så 