\chapter{Implementation}

\section{Preperation of Data}
The mri files are 256 $\times$ 256 $\times$ 256 matrices but we are only interested in the small part which overlaps with the masks from segmentation, i.e. where the elements in segmentation is either 1 (left hippocampus) or 2 (right hippocampus).
We have created the function HippoMatrix, which takes three variables, which file to load, wether or not erosion should be performed, and if the left or right hippocampus is desired.
First we assign a value based on if we are looking for the right hippocampus, as they are associeated with 1 and 2 respectfully.
If erosion is desired we create the city-block for erosion by taking advantage of distances calculations, distance = $\sqrt{x^2+y^2+z^2}$. All parts of the city-block have distance 1 from the origin point. With the city-block defined we can use matlabs build in function imerode to perform the erosion. \fixme[inline]{insert snip of code.+ link imerode page}

Looping through the entiere segmentation matrix we identify all the datapoints where segmentation is one for the left hippocampus or two if we are trying to identify the right hippocampus. For each instance in segmentation we save the coordinate (i,j,k) and the mri(i,j,k) value in an array as v(1) = (i$_1$,j$_1$,k$_1$,mri(i$_1$,j$_1$,k$_1$)).\fixme[inline]{snip of main loop, explain left is also used incase of right}

On the basis of this we can create a three-dimensional matrix which contains all the datapoints, hippoBox = max(i) - min(i) + 1 $\times$ max(j) - min(j) +1 $\times$ maj(k) - min(k) + 1.
Then we simply loop through our array with the relevant data and input them into hippoBox, all other elements inside the matrix are set to NaN. \fixme[inline]{snip of last loop}

The return value from the function is the matrix hippoBox containing only the relevant data.

\section{Calculating GLCMs}

To calculate the GLCMs in two-dimensions we have taken advantage of matlabs built-in function graycomatrix. It calculates as described in methods. It is then a matter of giving the proper offsets, and the right number of GIs. We can then loop through the hippoBox slices and sum up the GLCMs.\fixme[inline]{snip af glcm2dallangels hvor vi udregner}

We ultimately save all 90 glcms in a cell.\fixme[inline]{Fodnote der beksriver hvad en cell er?}

To implement the three-dimensional GLCMs we have created our own function.
The function GLCM3D takes a hippoBox as data and how many distances desired. I then for each distance loop through the entire matrix, and for each element it chekcs if it is NaN value and larger than zero. The check utilizes that NaN is not larger than zero, so data(i,j,k) > 0 returns false incase of data(i,j,k) = NaN.
The reason we also insist that it should also be larger than zero is because a few of the right hippocampus include GI of value zero in their hippoBox, but as zero is the value the mri scans have outside the brain we choose to ignore the few instances. To include them would mean we had to increase our GLCMs by 1 in size, which would make them differ from the GLCMs derived in two-dimensions, making the comparison unfair. In addition matlab start their index for their matrices with one and not zero so we would also have to add every index with one creating greater complexity.
\fixme[inline]{snip af første if.}
Given that the datapoint is relevant, i.e. larger than zero, we then have to look at the thirteen offsets, to see if we need to increment an element in on of the GLCMs. For each offset check if the offset is inside the hippobox, and is the offset elemnt a non NaN nonzero value. If so we then increment the relevant GLCM, lets say it is the offset(d,0,0), d = 1, in element GLCM$_(d,0,0)$(x,y) (hippoBox(i,j,k),hippoBox(i+1,j,k)).
\fixme[inline]{indsæt en if statement}


\section{Calculating(Computing) the GLCM Features}

In the implementation of the GLCM Feature derivation we are taking two inputs. The first input variable is the GLCM matrix and the second is wether we wish to normalize the data.

What we are doing first is to make sure that all variables are implemented. Firstly we find the size of the GLCM which will be the greylevels. Hereafter we can initiate the $C_x$, $C_y$, $C_{x+y}$ and $C_{x-y}$ since we know the size of the GLCM.

For the pixel values in the GLCM we are using MATLAB's \texttt{ind2sub} function, that is a command that determines the equivalent subscript values corresponding to a single index into an array.\fxnote[inline]{Lav et eksempel (diagram) af hvordan cxminusy og cxplusy ser ud}. We are using these variables in the GLCM Features as seen in Appendix \ref{derivationfeatures}.

To calculate the $C_{x+y}$ and $C_{x-y}$ we have two for loops as seen in Appendix \ref{Cxplusy} and \ref{Cxminusy} where N of course is the greylevels.

To find the mean and standard deviation for $C_x$ and $C_y$ we just use the functions that MATLAB have.

The GLCM features, as seen in Appendix \ref{derivationfeatures}, utilizes MATLABs use of vectorization. This is rewarding in the vectorized code appears more like the mathematical expressions and makes the code easier to understand and is shorter. There is often a performance gain in using vectorized code than the corresponding code containing loops.

It should be obvious for the reader to tell that the code looks alot like the mathematical expression like in Appendix \ref{derivationfeatures}.

\begin{lstlisting}[language=Matlab]
HXY1 = -nansum(glcm(tmpsub)'.*log(cX(I).*cY(J)));
HXY2 = -nansum(cX(I).*cY(J).*log(cX(I).*cY(J)));
HX   = -nansum(cX.*log(cX));
HY   = -nansum(cY.*log(cY));
HXY  = -nansum(glcm(:).*log(glcm(:)));

stats.angularSecondMoment                = sum(glcm(:).^2);
stats.contrast                           = sum(abs(I-J).^2.*glcm(tmpsub));
stats.correlation                        = (sum(I.*J.*glcm(tmpsub)) - muX*muY) ./ (stdX*stdY);
stats.variance                           = sum(((I - mean(glcm(:))).^2).*glcm(tmpsub));
stats.inverseDifferenceMoment            = sum(glcm(tmpsub)./(1 + (I-J).^2));
stats.sumAverage                         = sum(bsxfun(@times,(2:2*nGrayLevels)',cXplusY));
stats.sumVariance                        = sum(((2:2*nGrayLevels) - stats.sumAverage)'.^2.*cXplusY((2:2*nGrayLevels)-1,1));
stats.sumEntropy                         = nansum(cXplusY.*log(cXplusY));
stats.entropy                            = HXY;
stats.differenceVariance                 = var(cXminusY);
stats.differenceEntropy                  = nansum(cXminusY.*log(cXminusY));
stats.informationMeasuresOfCorrelation1  = (HXY - HXY1)./(max(HX,HY));
\end{lstlisting}


